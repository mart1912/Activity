\documentclass{article}
\usepackage[T2A]{fontenc}
\usepackage[utf8]{inputenc}
\usepackage{amsthm}
\usepackage{amsmath}
\usepackage{amssymb}
\usepackage{amsfonts}
\usepackage{mathrsfs}
\usepackage[12pt]{extsizes}
\usepackage{fancyvrb}
\usepackage{indentfirst}
\usepackage{parskip}
\setlength{\parindent}{1.5em}
\usepackage[
  left=2cm, right=2cm, top=2cm, bottom=2cm, headsep=0.2cm, footskip=0.6cm, bindingoffset=0cm
]{geometry}
\usepackage[english,russian]{babel}


\begin{document}
\section*{Вариант 19}
Упорядоченный группоид $(A,\cdot, \leqslant)$ тогда и только тогда принадлежит классу \textbf{R}$\{*, \subset\}$, когда для всякой $n$-диады $\omega = (\alpha, \beta)$ и любых термов $p_1,\dots,p_n$;\;$\tilde{p_0},\dots,\tilde{p_n}$ и $q_{i,j}$\;$(i,j = 0,\dots l_n)$ таких, что $G_{k-1}^{(\alpha(k), \beta(k))} \prec G(p_k)\;(k=1,\dots n)$, $G_\omega^{(r,r+1)} \prec G(\tilde{p}_t)\;(t = 0,\dots n;\;r = 0$ для $t = 0$ и $r = th + 1$ для $t = 1,\dots n)$ и $G_\omega^{(i,j)} \prec G(q_{i,j})\;(i,j = 0,\dots l_n)$, выполняются аксиомы:
\begin{center}
$\left(\displaystyle\bigwedge\limits^n_{k=0}x_{3k+1} \neq \textbf{0} \wedge \bigwedge\limits^n_{k=1} p_k \leqslant x_{3k-1}x_{3k}\right) \to x_{3t+1} \leqslant \tilde{p}_t,$

$\left(\displaystyle\bigwedge\limits^n_{k=0}x_{3k+1} \neq \textbf{0} \wedge \bigwedge\limits^n_{k=1} p_k \leqslant x_{3k-1}x_{3k}\right) \to q_{i,j} \neq \textbf{0},$
\end{center}
где \textbf{0} "--- нулевой элемент упорядоченного группоида $(A,\cdot,\leqslant)$.
\end{document}

