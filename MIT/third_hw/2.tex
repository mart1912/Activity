    Теперь все знают, как опасно глотать камни.

    Один даже мой знакомый сочинил такое выражение: <<Кавео>>, что значит: <<Камни  внутрь опасно>>.  И хорошо сделал. <<Кавео>> легко запомнить, и как потребуется,  так и вспомнишь сразу.

    А служил этот  мой  знакомый  истопником при паровозе. То по северной ветви  ездил, а то в Москву. Звали его Николай Иванович Серпухов, а курил он папиросы <<Ракета>>, 35 коп. коробка, и всегда говорил,  что  от  них  он меньше кашлем страдает,  а от  пятирублевых, говорит, я всегда задыхаюсь.

    И  вот случилось однажды Николаю  Ивановичу попасть в Европейскую гостиницу,  в ресторан. Сидит Николай Иванович за  столиком, а за соседним столиком  иностранцы  сидят  и яблоки жрут.

    Вот  тут"=то  Николай Иванович  и  сказал себе: <<Интересно, \ "--~ \ сказал себе Николай Иванович, "--* как человек устроен>>.

    Только  это он  себе сказал,  откуда  ни возьмись, появляется перед  ним фея и  говорит:

    "--* Чего тебе, добрый человек, нужно?
    
    Ну, конечно, в ресторане происходит движение, откуда, мол, эта неизвестная  дамочка возникла.  Иностранцы  так даже яблоки жрать перестали.

    Николай"=то  Иванович  и сам не на  шутку струхнул  и говорит просто так,  чтобы отвязаться:

    "--* Извините, \ "--~ \ говорит, "--*  особого такого ничего мне не требуется.

    "--* Нет, \ "--~ \  говорит неизвестная дамочка, "--* я, \ "--~ \ говорит,  "--*  что называется фея.  Одним моментом что угодно смастерю.

    Только видит Николай Иванович,  что  какой"=то гражданин  в серой паре внимательно к их разговору  прислушивается. А  в  открытые двери метродотель бежит, а за ним еще какой"=то субъект с папироской во рту.

    <<Что за черт! \ "--~ \ думает Николай Иванович, "--* неизвестно что получается>>.

    А оно и действительно неизвестно что получается. Метродотель по столам скачет, иностранцы ковры в трубочку закатывают, и вообще черт его знает! Кто во что горазд!

    Выбежал Николай Иванович на улицу,  даже шапку в раздевалке  из хранения не взял, выбежал на улицу Лассаля и сказал себе: <<Ка ве О! Камни внутрь опасно!  И чего"=чего  только на свете не бывает!>>

    А придя домой, Николай Иванович так сказал жене своей:

    "--* Не пугайтесь, Екатерина Петровна, и не волнуйтесь. Только нет  в мире никакого равновесия. И ошибка"=то  всего  на какие"=нибудь полтора килограмма на  всю вселенную,  а все же удивительно, Екатерина  Петровна,  совершенно удивительно!

    ВСЕ.
    
\begin{flushright}
18 сентября 1934 года.
\end{flushright}