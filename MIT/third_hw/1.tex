    Отвечает один другому:
    
    "--* Не видал я их.
    
    "--* Как же ты их не видал, \ "--~ \ говорит другой, "--* когда сам же на них шапки надевал?

    "--* А вот, \ "--~ \ говорит один, "--* шапки на них надевал, а их не видал.

    "--* Да возможно ли это? \ "--~ \ Говорит другой с длинными усами.

    "--* Да, \ "--~ \ говорит первый, "--* возможно, \ "--~ \ и улыбается синим ртом.

    Тогда другой, который с длинными усами, пристает к синерожему, чтобы тот объяснил ему, как это так возможно "--- шапки на людей надеть, а самих людей не заметить. А синерожий отказывается объяснять усатому, и качает своей головой, и усмехается своим синим ртом.

    "--* Ах ты дьявол ты этакий, \ "--~ \ говорит ему усатый. "--* Морочишь ты меня старика! Отвечай мне и не заворачивай мне мозги: видел ты их или не видел?

    Усмехнулся еще раз другой, который синерожий, и вдруг исчез, только одна шапка осталась в воздухе висеть.

    "--* Ах так вот кто ты такой! \ "--~ \ сказал усатый старик и протянул руку за шапкой, а шапка от руки в сторону. Старик за шапкой, а шапка от него, не дается в руки старику.

    Летит шапка по Некрасовской улице мимо булочной, мимо бань. Из пивной народ выбегает, на шапку с удивлением смотрит и обратно в пивную уходит.

    А старик бежит за шапкой, руки вперед вытянул, рот открыл; глаза у старика остеклянели, усы болтаются, а волосы перьями торчат во все стороны.

    Добежал  старик до Литейной,  а  там ему наперерез уже милиционер бежит и еще  какой"=то гражданин в сером костюмчике.  Схватили они безумного старика и повели его куда"=то.
    
\begin{flushright}
21 июля 1938 года.
\end{flushright}
