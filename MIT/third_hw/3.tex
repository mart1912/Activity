    Жил"=был  человек,  звали  его  Кузнецов. Однажды сломалась у него табуретка. Он вышел из дома и пошел в магазин купить  столярного клея, чтобы склеить табуретку.

    Когда Кузнецов проходил мимо недостроенного дома, сверху упал кирпич и ударил  Кузнецова по голове.

    Кузнецов  упал,  но  сразу же вскочил на ноги и пощупал свою голову. На голове у Кузнецова вскочила огромная шишка.

    Кузнецов погладил шишку рукой и сказал:
    
    "--* Я  гражданин Кузнецов, вышел из дома и пошел  в магазин, чтобы\dots\ чтобы\dots\ чтобы\dots\ Ах, что же это такое! Я забыл, зачем я пошел в магазин!

    В это время с крыши упал второй кирпич и опять стукнул Кузнецова по голове.

    "--* Ах! \ "--~ \ вскрикнул Кузнецов, схватился за голову и нащупал на голове вторую шишку.

    "--* Вот так история!  \ "--~ \  сказал  Кузнецов. "--* Я гражданин Кузнецов, вышел из дома и  пошел в\dots\ пошел  в\dots\ пошел в\dots\ куда же  я пошел? Я забыл, куда я пошел!

    Тут сверху на Кузнецова упал третий кирпич.  И на голове  Кузнецова вскочила третья шишка.

    "--* Ай"=ай"=ай! \ "--~ \  закричал Кузнецов, хватаясь за голову. "--*  Я гражданин Кузнецов вышел из\dots\ вышел из\dots\ вышел из погреба? Нет. Вышел из бочки? Нет! Откуда же я вышел?

    С  крыши  упал четвертый кирпич,  ударил Кузнецова  по затылку, и на затылке у Кузнецова вскочила четвертая шишка.

    "--* Ну и ну! \ "--~ \  сказал Кузнецов, почесывая затылок. "--* Я\dots\ я\dots\ я\dots\ Кто же я?  Никак я забыл, как меня зовут?  Вот так история! Как же меня зовут? Василий Петухов? Нет. Николай Сапогов? Нет. Пантелей Рысаков?  Нет. Ну кто же я?

    Но тут с  крыши  упал пятый кирпич и так стукнул Кузнецова по  затылку,  что Кузнецов окончательно позабыл  все на свете и крикнув <<О"=го"=го!>>, побежал по улице.

\begin {center}
*  *  *
\end{center}
                  
    Пожалуйста!  Если кто"=нибудь встретит на улице  человека, у  которого  на голове пять шишек, то  напомните ему, что зовут его Кузнецов и что ему нужно купить столярного клея и починить ломаную табуретку.

\begin{flushright}
1 ноября 1935 года.
\end{flushright}